\chapter{Network Topologies \& Models}

\section{Network Topology}

Network topology refers to the physical or logical layout of devices in a computer network. It determines how devices are interconnected and how data flows between them. Understanding topology is crucial for network design, performance, and maintenance.

\subsection{Bus Topology}
In a bus topology, all devices are connected to a single central cable, known as the bus or backbone. Data travels in both directions along the cable.

\begin{table}[H]
\centering
\caption{Bus Topology: Advantages and Disadvantages}
\begin{tabularx}{\linewidth}{|X|X|}
\hline
\textbf{Advantages} & \textbf{Disadvantages} \\
\hline
Easy to set up and requires less cable & A fault in the main cable can bring down the entire network \\
\hline
Cost-effective for small networks & Performance degrades as more devices are added \\
\hline
Ideal for temporary networks & Limited cable length and number of devices \\
\hline
\end{tabularx}
\end{table}

\subsection{Star Topology}
In a star topology, all devices are connected to a central hub or switch. Communication passes through this central node.

\begin{table}[H]
\centering
\caption{Star Topology: Advantages and Disadvantages}
\begin{tabularx}{\linewidth}{|X|X|}
\hline
\textbf{Advantages} & \textbf{Disadvantages} \\
\hline
Easy to manage and troubleshoot & If the central hub fails, the entire network goes down \\
\hline
Scalable – new devices can be added easily & Requires more cable than bus or ring topology \\
\hline
High performance under light to medium traffic & Central device adds cost \\
\hline
\end{tabularx}
\end{table}

\subsection{Ring Topology}
In a ring topology, devices are connected in a circular loop. Each device has two neighbors, and data travels in one direction.

\begin{table}[H]
\centering
\caption{Ring Topology: Advantages and Disadvantages}
\begin{tabularx}{\linewidth}{|X|X|}
\hline
\textbf{Advantages} & \textbf{Disadvantages} \\
\hline
Equal access for all devices & A break in the ring can disrupt the whole network \\
\hline
Performs better than bus under heavy traffic & Troubleshooting can be difficult \\
\hline
Efficient for handling predictable traffic flow & Adding/removing devices can disrupt network \\
\hline
\end{tabularx}
\end{table}

\subsection{Mesh Topology}
In a mesh topology, every device is connected to every other device. It can be full or partial mesh.

\begin{table}[H]
\centering
\caption{Mesh Topology: Advantages and Disadvantages}
\begin{tabularx}{\linewidth}{|X|X|}
\hline
\textbf{Advantages} & \textbf{Disadvantages} \\
\hline
Highly reliable and fault-tolerant & Requires a lot of cabling and ports \\
\hline
Data can be routed via multiple paths & Complex and expensive to set up \\
\hline
Suitable for critical systems (e.g., military, banking) & Difficult to maintain in large networks \\
\hline
\end{tabularx}
\end{table}

\subsection{Tree Topology}
Tree topology is a combination of star and bus topologies. It has a hierarchical layout with a root node and branching segments.

\begin{table}[H]
\centering
\caption{Tree Topology: Advantages and Disadvantages}
\begin{tabularx}{\linewidth}{|X|X|}
\hline
\textbf{Advantages} & \textbf{Disadvantages} \\
\hline
Supports scalability and easy expansion & Dependent on the backbone line \\
\hline
Well-suited for large organizations & More cabling and configuration needed \\
\hline
Allows for segmentation and management & Troubleshooting can be complex \\
\hline
\end{tabularx}
\end{table}

\subsection{Hybrid Topology}
Hybrid topology combines two or more different topologies to leverage their strengths and minimize weaknesses.

\begin{table}[H]
\centering
\caption{Hybrid Topology: Advantages and Disadvantages}
\begin{tabularx}{\linewidth}{|X|X|}
\hline
\textbf{Advantages} & \textbf{Disadvantages} \\
\hline
Highly flexible and customizable & Complex design and management \\
\hline
Scalable for large networks & Higher implementation cost \\
\hline
Combines benefits of multiple topologies & Configuration and maintenance can be challenging \\
\hline
\end{tabularx}
\end{table}

\section{OSI Model}


\section{TCP/IP Model}

\section{Comparison between OSI \& TCP/IP}